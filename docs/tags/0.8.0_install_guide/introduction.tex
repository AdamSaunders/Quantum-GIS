% vim:textwidth=76:autoindent

\section{Introduction}\label{label_introduction}
\pagenumbering{arabic}
\setcounter{page}{1}

The majority of this document is devoted to describing how to build
QGIS~\CURRENT
(\textit{`TITAN'}) from the source distribution. These instructions are for
Linux/Unix and other POSIX systems which have the required build
environment. If you are building on FreeBSD, see
\url{http://qgis.org/index.php?option=com_content&task=view&id=84&Itemid=86} for hints and further
information.

Installing on Windows and Mac OS X is a simple process as described
below.

You don't have to build all the QGIS dependencies from source. If your
platform provides packages at an acceptable version for the needed
dependencies, you can install them prior to building QGIS. Make sure you
also install the "development" package (if separate from the main package)
for each dependency. QGIS needs the header files from these packages in
order to build. 

The latest version of this document can always be found at 
\url{http://qgis.org/index.php?option=com_content&task=view&id=106&Itemid=89}.
