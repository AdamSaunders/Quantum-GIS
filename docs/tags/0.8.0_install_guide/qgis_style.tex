%% Definitions for html output
%% [$Id$]
%% nothing to change below

% important for nice acroread-visualization as pdf.
\usepackage[latin1]{inputenc}
\usepackage[T1]{fontenc} 

% To make all GRASS commands appear smooth
\usepackage{ae,aecompl}

% Change of Caption font and size
\usepackage[bf]{caption2}
\renewcommand{\captionfont}{\small}

% http://irb.cs.tu-berlin.de/leitfaden/manuale/latex-local/html/node20.html
% Include HTML package for latex2html
\usepackage{html}

% Serif font
%\usepackage{palatino}	
% mathpazo are mathefonts for formulars in palatino
%\usepackage{mathpazo}

% change serif to helvetica
\usepackage{helvet}
\renewcommand{\sfdefault}{phv}  % switch \sf to Helvetica T1
\renewcommand*{\familydefault}{\sfdefault} 

% new command to suppress appendix captions
\def\afterfi#1\fi{\fi#1}
\let\ORIGaddtocontents\addtocontents
\newcommand*\dontaddtolof[2]{\edef\temp{#1}%
  \ifx\temp\ext@figure\else\afterfi\ORIGaddtocontents{#1}{#2}\fi}
\newcommand*\ignorelof{\let\addtocontents\dontaddtolof}
\newcommand*\obeylof{\let\addtocontents\ORIGaddtocontents}

% /usr/share/doc/texmf/latex/graphics/grfguide.ps.gz for infos, how 
% to define graphicx parameter.
% graficx-package for dvips and dvipdf and includegraphics 
\usepackage[dvips,dvipdf]{graphicx}
	\graphicspath{{./finalpix/}{../finalpix/}}

% color package for dvips
\usepackage[dvips]{color}

%author-year-citation paket natbib (needed for literature index)
\usepackage[round,sort]{natbib}		
%  \bibpunct{(}{)}{,}{a}{}{;}

%not needed for US letter:
% page width and height: US letter
%\usepackage{typearea}                    
%   \areaset[12mm]{8.5in}{11in}

% A4:
\usepackage{typearea}                    
   \areaset[12mm]{17cm}{22cm}
\topmargin-0.2cm % A4

% For floating texts around images and tables
\usepackage{floatflt}

% to allow hyphenation for re-projection (composita)
\usepackage{hyphenat}

% header and footer package
\usepackage{fancyhdr}
% \setlength{\unitlength}{1mm}
\pagestyle{fancy}
\fancyhead{} %clear all fields
\fancyfoot{} %clear all fields

% \rhead{\it{\nouppercase{\leftmark}}}
% \chead{}
% \lhead{}
% \lfoot{QGIS Installation Guide}
% \cfoot{}
% \rfoot{\hbox{}\hfill\thepage}                               
% \renewcommand{\footrulewidth}{0.3pt}		%line width footer
% \renewcommand{\headrulewidth}{0.3pt}           %line width header

%% fancy positions
%% +-------------+-----------------+
%% | LE  CE   RE | LO    CO     RO |   <- head
%% |  even  left | odd page  right |
%% | LE  CE   RE | LO    CO     RO |   <- foot
%% +-------------+-----------------+

\fancyhead[RO]{\rightmark}  % print section right on odd
\fancyhead[LE]{\leftmark}   % print chapter left on even page
\fancyhead[CE,RE]{}         % empty
\fancyhead[LO,CO]{}         % empty
%double sided footer:
%\fancyfoot[LE,RO]{\thepage}
%\fancyfoot[RE,LO]{QGIS Installation Guide}
%single sided footer:
\fancyfoot[RE,RO]{\thepage} % pages in footer, single sided
\fancyfoot[LE,LO]{QGIS Installation Guide}

\renewcommand{\headrulewidth}{0.3pt}  %  ruler at top
\renewcommand{\footrulewidth}{0.3pt}  %  ruler at bottom

%define plain for chapter's first page:
\fancypagestyle{plain}{%
\fancyhf{}                 %clear all fields
\fancyfoot[C]{\thepage}    %put page into center footer
\renewcommand{\headrulewidth}{0pt}  % no ruler at top
\renewcommand{\footrulewidth}{0pt}} % no ruler at bottom

\pagenumbering{roman}
 
% for extra long tables
\usepackage{supertabular}

% create an automatic index
\usepackage{makeidx}

% for page-width tables
\usepackage{tabularx}

% rotate tables
\usepackage{rotating}

% table rows and cols with grey background color
\usepackage{colortbl}

% don't cover column lines
\usepackage{hhline}

% footer formating
\usepackage[bottom]{footmisc}

% row distance
\renewcommand{\baselinestretch}{1.1}             
\small\normalsize

% sloppy syllable division
\sloppy

% paragraph distance
\parskip1.5ex              

%  at begining of a new sentence, here 0
\parindent0ex

% also in article-style 5 structure level!
% 3 are enough... to keep it clear
\setcounter{secnumdepth}{3}
\setcounter{tocdepth}{3}
% new definition for paragraph und subparagraph 
% paragraph
\renewcommand\paragraph{\@startsection{paragraph}{4}{\z@}%
				{-3.25ex\@plus -1ex \@minus -.2ex}%
				{1ex \@plus .2ex}%
			{\normalfont\normalsize\bfseries}}
% subparagraph
\renewcommand\subparagraph{\@startsection{subparagraph}{5}{\z@}%
				{-3.25ex\@plus -1ex \@minus -.2ex}%
				{1ex \@plus .2ex}%
			{\normalfont\normalsize\bfseries}}

% several graphics next to eachother counting  
\usepackage[normal]{subfigure}

% new command for subfigure to optimize distance between figures
\newcommand{\goodgap}{%
 \hspace{\subfigtopskip}%
 \hspace{\subfigbottomskip}}

% keep float-grafics in a section
\usepackage[below]{placeins}

% map index
\usepackage{float}
% generate new map index.
\newfloat{karte}{H}{maps}
 \floatstyle{plain}
 \floatname{karte}{Karte}

% command for subscript and superscript in formulars.
\newfont{\tensy}{cmsy10}
\newcommand{\hoch}[1]{{$\fontdimen16\tensy=3.0pt
                     \fontdimen17\tensy=3.0pt \mathrm{#1}$}}

% For 2 separated bib indices (lit and web)
\usepackage{bibtopic}

% Fonts for all QGIS/Unix commands
\newcommand{\qgis}{\texttt}

% using xspace
\usepackage{xspace}

% Macros QGIS versions
\def\CURRENT{0.8\xspace}
\def\OLD{0.7\xspace}

% Script version
\def\SKRIPTVERSION{1.0\xspace}

% QGIS version:
\def\QGISVERSION{0.8\xspace}

% release month:
\def\RELEASEMONTH{January 2006\xspace}


\usepackage[bookmarks=true, bookmarksnumbered=false, pdftitle={QGIS Documentation}, pdfauthor={QGIS Project}, pdfsubject={QGIS Installation Guide}, breaklinks=true, colorlinks=false, linkcolor=black,dvips]{hyperref}
%bei colorlinks=false sollte man pdfborder={0 0 0} reinmachen (keine link-Boxen)

%%%%%%%%%%%%%%%%%%%%%%%%%%%%%%%%%%%%%%%%%

