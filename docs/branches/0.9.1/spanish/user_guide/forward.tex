% vim: set textwidth=78 autoindent:

\section{Prefacio}\label{label_forward}
\pagenumbering{arabic}
\setcounter{page}{1}

Bienvenido al maravilloso mundo de los Sistemas de Información Geográfica(SIG)!
Quantum GIS (QGIS) es un Sistema de Información Geográfica de Código Abierto. 
El proyecto nació en Mayo de 2002 y se estableció como un proyecto dentro de 
SourceForge en Junio del mismo año. Hemos trabajado duro para hacer del 
software SIG (que tradicionalmente es software comercial caro) una posibilidad
viable para cualquiera con un acceso básico a un ordenador personal. 
Actualmente QGIS corre en la mayoría de plataformas Unix, Windows, y OS X. 
QGIS está desarrollado utilizando el Qt toolkit (\url{http://www.trolltech.com})
y C++. Esto hace que QGIS sea rápido y tenga una interfaz de usuario agradable
y fácil de usar. 

QGIS espera ser un SIG fácil de usar, proporcionando características y 
funciones comunes. El objetivo incial fue proporcionar un visor de datos SIG. 
QGIS ha alcanzado este punto en su evolución y se está utilizando por muchos 
para sus necesidades diarias de visualización de datos SIG. QGIS soporta un 
buen número de formatos raster y vectoriales, con nuevos soportes fácilmente
añadidos utilizando su arquitectura de plugins (ver Apendice \ref{appdx_data_formats} 
para consultar la lista completa de los formatos de datos soportados).
QGIS se ha publicado bajo Licencia Pública (GNU General Public License) (GPL). 
Desarrollar QGIS bajo esta licencia quiere decir que puedes inspeccionar y 
modificar el código fuente y las garantías que tienes, nuestros felices 
usuarios siempre tienen acceso a un programa SIG gratis y que puede ser 
libremente modificado. Debes tener una copia de la licencia con tu copia 
de QGIS, y tampién puedes encontrala como Apendice \ref{gpl_appendix}.  

\begin{quote}
\begin{center}
\textbf{Note:} La última versión de este documento siempre se ecuentra en \newline
http://qgis.org/docs/userguide.pdf 
\end{center}
\end{quote}

\subsection{Características}\label{label_majfeat}

QGIS tiene muchas funciones y características comunes a todos los SIG. 
Las características princiaples se enumeran aqui debajo, divididas en
elementos del núcleo y Plugins. \\

\textbf{Elementos del Núcleo}

\begin{itemize}
\item Soporte raster y vectorial mediante la librería OGR
\item Soporte para PostgreSQL con tablas espaciales utilizando PostGIS
\item Integración con GRASS, incluída visualización, edición y análisis
\item Digitalización GRASS y OGR/Shapefile
\item Composición de Mapas
\item Soporte OGC
\item Panel de Vista General
\item Marcadores espaciales
\item Identificar/Seleccionar elementos
\item Editar/Visualizar/Buscar atributos
\item Etiquetado de elementos
\item Proyecciones al vuelo
\item Salvar y recuperar proyectos
\item Exportar ficheros map a Mapserver
\item Cambiar simbología vectorial y raster 
\item Arquitectura extensible con plugins
\end{itemize}

\textbf{Plugins}

\begin{itemize}
\item Añadir capas WFS
\item Añadir capas de texto delimitadas
\item Decoración (Etiqueta Copyright, Flecha de Norte y Barra de Escala)
\item Georreferencación
\item Herramientas GPS
\item GRASS
\item Generador de mallas
\item Funciones de Geoprocesamiento PostgreSQL
\item Herramienta de importación de SPIT Shapefile a PostgreSQL/PostGIS 
\item Consola Python
\item openModeller
\end{itemize}

\subsection{Whats New in 0.9}\label{label_whatsnew}

Version 0.9.0 brought some very interesting new features to you.

\begin{itemize}
\item El lenguaje Python posibilita escribir plugins en Python y crear 
aplicaciones SIG que utilicen librerías de QGIS
\item Eliminado``automake build system`` - QGIS ahora necesita CMake para su 
compilación
\item Algunos modulos nuevos GRASS añadidos a la barra de herramientas
\item Actualizaciones en el editor de Mapas
\item Correcciones en los ficheros shape 2.5D 
\item Mejoras en la Georeferenciación
\item Soporte de localización extendido a 26 lenguajes    
\end{itemize}

QGIS \CURRENT concentrates on stabilization and feature enhancement.

\begin{itemize}
\item 66 bugfixes and feature improvements 
\item New window arrangement feature for the Georeferencer
\item New locale tab in the options dialog
\item Download progress information for WMS and WFS data
\item More GRASS modules added to the GRASS toolbox
\end{itemize}

