% vim: set textwidth=78 autoindent:

\section{Introducción a los SIG}\label{label_intro}

Un Sistema de Información Geográfica (SIG)\cite{mitchel05}\footnote{Este 
capítulo es de Tyler Mitchell (\url{http://www.oreillynet.com/pub/wlg/7053}) y 
se utiliza bajo Licencia Creative Commons. Tyler es el autor de 
\textit{Web Mapping Illustrated}, publicado por O'Reilly, 2005.}
es una colección de software que te permite crear, visualizar, consultar y 
analizar datos geoespaciales. Los datos geoespaciales se refieren a 
información sobre la localización geográfica de una entidad. Esto 
habitualmente implica el uso de una coordenada geográfica, como los valores 
latitud o longitud. Otros terminos de datos espaciales comúnmente utilizados 
son: datos geográficos, datos SIG/GIS, datos de mapa, datos de localización, 
coordenadas y datos geométricos espaciales.

Las aplicaciones que utilizan datos geoespaciales realizan gran variedad de 
funciones. La producción de Mapas es la función más fácil de entender de 
las aplicaciones geoespaciales. Los programas de edición de mapas cogen datos 
geoespaciales y los transforman en datos visibles, normalmente sobre pantallas
de ordenador o páginas impresas.
Las aplicaciones pueden presentar mapas estáticos (una simple imagen) o mapas
dinámicos que están personalizados por quien esté viendo el mapa a través de 
un programa de escritorio o una página web.

Mucha gente cree erroneamente que las aplicaciones geoespaciales sólo producen
mapas, pero el análisis de datos geoespaciales es otra de las principales 
funciones de estas aplicaciones. Algunos tipos de análisis típicos incluyen
computación:

\begin{enumerate}
\item distancias entre localizaciones geográficas
\item la cantidad de área (p.ej., metros cuadrados) dentro de cierta región
geográfica
\item qué elementos geográficos solapan otros elementos
\item la cantidad de solapes entre elementos
\item el número de localizaciones dentro de cierta distancia
\item y cosas así...
\end{enumerate}

Esto puede parecer simplista, pero pueden ser aplicadas en todos los tipos de
vías a través de muchas disciplinas. El resultado de un análisis se puede 
mostrar sobre un mapa, pero habitualmente se tabula en un informe para ayuda 
en la toma de decisiones.

El reciente fenómemo de los servicios basados en localización promete 
introducir todo tipo de características, pero muchas estarán basadas en una
combinación de mapas y análisis. Por ejemplo, puedes tener un teléfono móvil
que deje rastro de tu localización geográfica. Si tienes el software adecuado
el teléfono puede decirte que clase de restaurantes están a corta distancia. 
A pesar de que es una aplicación novedosa de tecnología geoespacial, 
básicamente está haciendo análisis geoespacial y lista los resultados para ti.

\subsection{¿Por qué todo esto es tan nuevo?}\label{label_whynew}

Bien, no todo. Hay muchos dispositivos hardware nuevos que están posibilitando
los servicios móviles geoespaciales. También están disponibles muchas 
aplicaciones geoespaciales de código abierto, pero la existencia hardware y 
software especializado en la industria no es nada nuevo.
Los sistemas de posicionamiento global (GPS) son frecuentes, pero llevan
utilizándose en la industria hace más de una decada. También, las herramientas
de edición de mapas y análisis han sido un importante mercado, principalmente 
orientado a industrias de gestión de rescursos naturales.

Qué es nuevo, es sobre cómo el último hardware y software está siendo aplicado
y quién lo esta aplicando. Los usuarios tradicionales de herramientas de 
edición de mapas y análisis eran analistas SIG fuertemente entrenados o 
técnicos de edición de mapas entrenados para utilizar herramientas CAD. 
Ahora, las capacidades de procesamiento de un PC doméstico y paquetes software 
de código abierto han descubierto un arma para aficionados, profesionales,
desarrolladores web, etc. para interactuar con datos geoespaciales. La curva 
de aprendizaje está bajando. El costo está bajando. La cantidad de saturación 
de tecnología geoespacial se está incrementando.

¿Cómo se almacenan los datos geoespaciales? En pocas palabras, hay dos tipos
de datos geoespaciales ampliamente utilizados hoy. En adición a los datos 
tabulares tradicionales que también son ampliamente utilizados por 
aplicaciones geoespaciales.

\subsubsection{Datos Raster}\label{label_rasterdata}

Un tipo de dato geoespacial se llama dato raster o simplemente ''un raster''. 
La forma más fácilmente reconocible de un raster son las imágenes de satélite
o fotos aéreas. Los modelos de elevación o de sombras también se representan 
típicamente como un raster. Cualquier tipo de elemento de un mapa se puede 
representar como un raster, pero tienen algunas limitaciones.

Un raster es una malla regular formada por celdas, o en su caso imágenes, 
pixeles. Tienen un número fijo de filas y columnas. Cada celda tiene un valor 
numérico y cierto tamaño geográfico (p.ej. un tamaño de 30x30 metros).

Se utilizan múltiples raster superpuestos para representar imágenes con más 
colores (p.ej. un raster por cada valor de rojo, verde y azul se combinan para 
crear una imágen a color). Las imágenes de satelite también representan datos 
en multiples ''bandas''. Cada banda es esencialmente un raster individual,
espacialmente superpuesto donde cada banda mantiene valores de cierta longitud
de onda. Como puedes imaginar, un raster grande, ocupa más espacio en disco. 
Un raster con celdas más pequeñas proporciona más detalle, pero ocupa más 
espacio en disco. El truco es encontrar el balance correcto entre el tamaño 
de celda para almacenar y el tamaño de celda para análisis o mapas.

\subsubsection{Datos Vectoriales}\label{label_vectordata}

Los datos vectoriales también se utilizan en aplicaciones geoespaciales. 
Si estuviste atento durante las clases de trigonometría y coordenadas 
geográficas, estarás familiarizado con algunas de las cualidades de los datos 
vectoriales. En su sentido más simple, los vectores son una vía para describir 
una localización utilizando un conjunto de coordenadas. Cada coordenada hace 
referencia a una localización geográfica mediante un sistema de valores x e y.

Esto puede parecerse a un plano cartesiano - ya sabes, los diagramas de la 
escuela que mostraban ejes x,y. Los puedes haber utilizado en gráficas para 
ver los ahorros de tu jubilación o para ver el incremento del interés de la 
hipoteca, pero el concepto es el mismo para el análisis espacial y la edición 
de mapas.

Hay formas obvias de representar estas coordenadas geográficas dependiendo de 
tus propositos. Esto es un área completa de estudio para otro día 
-proyecciones de mapas.

Los datos vectoriales son de tres formas:

\begin{enumerate}
\item Puntos - Una única coordenada (x y) representa una localización 
geográfica discreta
\item Líneas - Múltiples coordenadas (x1 y1, x2 y2, x3 y4, ... xn yn) unidas 
en cierto orden. Como dibujar una línea de punto (x1 y1) a punto (x2 y2). 
Las partes entre cada punto se consideran segmentos. Estos tienen una longitud 
y la línea puede indicar la dirección base del orden de los puntos. 
Técnicamente, una línea es un par de coordenadas conectadas; mientras que un 
segmento múltiple son múltiples líneas conectadas juntas.  
\item Polígonos - Cuando las líneas están unidas por más de dos puntos, con el 
último punto coincidiendo con el primero, lo podemos llamar polígono. 
Un triángulo, circulo, rectángulo, etc. son todos polígonos. La característica 
clave de los polígonos es que tienen un área dentro de ellos.  
\end{enumerate}

