% vim: set textwidth=78 autoindent:
\chapter{Forward}

Welcome to the wonderful world of Geographical Information Systems (GIS)!
Quantum GIS (QGIS) is an Open Source Geographic Information System. The project
was born in May of 2002 and was established as a project on SourceForge in June
of the same year. We've worked hard to make GIS software (which is traditionaly
expensive commerical software) a viable prospect for anyone with basic access
to a Personal Computer. QGIS currently runs on most Unix platforms, Windows, and
OS X. QGIS is developed using the Qt toolkit (\url{http://www.trolltech.com})
and C++. This means that QGIS feels snappy to use and has a pleasing, easy to
use graphical user interface. 

QGIS aims to be an easy to use GIS, providing common functions and features.
The initial goal was to provide a GIS data viewer. QGIS has reached that point
in its evolution and is being used by many for their daily GIS data viewing
needs. QGIS supports a number of raster and vector data formats, with new
support easily added using the plugin architecture (see Appendix
\ref{appdx_data_formats} for a full list of currenly supported data formats).
QGIS is released under the GNU Public License (GPL). Developing QGIS under this
license means that you can (if you want to) inspect and modify the source code
and guarantees that you, our happy user will always have access to a GIS
program that is free of cost and can be freely modified. You should have
received a full copy of the license with your copy of QGIS, and is also
available as Appendix \ref{gpl_appendix}.  
\begin{quote}
\begin{center}
\textbf{Note:} The latest version of this document can always be found at \newline
http://qgis.sourceforge.net/docs/userguide.html 
\end{center}
\end{quote}

\section{Major Features}

QGIS has many common GIS features and functions. The major features
are listed below. 

\begin{compactenum}
\item Support for spatially enabled PostgreSQL tables using PostGIS 
\item Support for ESRI shapefiles and other vector formats support by the
OGR library, including MapInfo files 
\item GRASS integration, including view, edit, and analysis
\item On the fly projection of vector layers
\item Map composer
\item Identify features 
\item Display attribute table 
\item Select features 
\item Label features
\item Persistent selections 
\item Save and restore projects
\item Support for raster formats supported by the GDAL library 
\item Change vector symbology (single, graduated, unique value, and continuous) 
\item SVG markers symbology (single, unique value, and graduated) 
\item Display raster data such as digital elevation models, aerial photography
or landsat imagery 
\item Change raster symbology (grayscale, pseudocolor and multiband RGB) 
\item Export to Mapserver map file 
\item Digitizing support
\item Map overview
\item Plugins 
\end{compactenum}


\section{Whats New in 0.7}
Version 0.7 brings several important features, including projection support, a map composer, and better integration with GRASS. The major new features in this release include:
\begin{compactenum}
  \item On the fly projection for reprojecting layers in different coordinate systems
  \item Map Composer for creating print layouts
  \item Toolbox for running GRASS tools from QGIS
  \item Raster graphing tool to produce a histogram for a raster layer.
  \item Raster query using the identify tool to get the pixel
  values from a raster 
  \item New customizable settings for the digitizing line width, color, and selection color  
  \item New symbols for use with point layers are available from the layer properties dialog 
  \item Spatial bookmarks allow you to create and manage bookmarks for an area on the map
  \item Measure tool to measure distances on the map with both
  segment length and total length displayed as you click
  \item GPX loading times and memory consumption for large GPX (GPS) files
  has been drastically reduced.  
  \item Digitizing enhancements, including the ability to capture data straight
  into PostgreSQL/PostGIS, and improvements to the definition of attribute tables
  for newly created layers.
  \item Raster Georeferencing plugin can be used
  to generate a world file for a raster by defining known
  control points in the raster coordinate system.
\end{compactenum}

