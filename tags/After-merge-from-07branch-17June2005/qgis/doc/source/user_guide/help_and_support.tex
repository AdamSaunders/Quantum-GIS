\chapter{Help and Support}

QGIS is still under active development and as such it won't always work like you expect it to.
The preferred way to get help is by joining the qgis-users mailing list Your questions will reach abroader audience and answers will benefit others. You can subscribe to the qgis-users mailing list by visiting here: http://lists.sourceforge.net/lists/listinfo/qgis-user

If you are a developer facing problems of a more technical nature, you may want to join the qgis-developer mailing list here: http://lists.sourceforge.net/lists/listinfo/qgis-developer

We also maintain a presence on IRC - visit us by joining the \#qgis channel on irc.freenode.net. Please wait around for a response to your question as many folks on the channel are doing other things andit may take a while for them to notice your question.
Commercial support for QGIS is available from Micro Resources 

While the qgis-users mailing list is useful for general 'how do I do xyz in QGIS' type questions, you may wish to notify us about bugs in QGIS. You can submit bug reports using the QGIS bug tracker. When reporting a bug, either login to SourceForge or, if you don't have a SourceForge id, provide an email address where we can request additional information.
Feature requests can be submitted using the feature tracker. Please bear in mind that your bug may not always enjoy the priority you might hope for (depending on its severity). Some bugs may require may require significant developer effort to remedy and the manpower is not always available for this.

If you have found a bug and fixed it yourself you can submit it to the QGIS Sourceforge patch queue where someone will review it and apply it to QGIS. Please dont be alarmed if your patch is not applied straight away - developers may be tied up with other committments.

There is also a community site for QGIS where we encourage QGIS users to share their experiences and provide case studies about how they are using QGIS. The community site is available at: http://community.qgis.org 

Lastly, we maintain a WIKI web site at http://wiki.qgis.org where you can find a variety of useful information relating to QGIS development, release plans, links to download sites and so on.